\documentclass[conference]{lib/IEEEtran}
\IEEEoverridecommandlockouts
% The preceding line is only needed to identify funding in the first footnote. If that is unneeded, please comment it out.
%Template version as of 6/27/2024

\usepackage{lib/cite}
\usepackage{amsmath,amssymb,amsfonts}
\usepackage{algorithmic}
\usepackage{graphicx}
\usepackage{textcomp}
\usepackage{lib/xcolor}
\def\BibTeX{{\rm B\kern-.05em{\sc i\kern-.025em b}\kern-.08em
    T\kern-.1667em\lower.7ex\hbox{E}\kern-.125emX}}
\begin{document}

\title{Accelerometer and Gyroscope based IoT System for Activity Classification\\
}

\author{\IEEEauthorblockN{Séamus Knightly}
\IEEEauthorblockA{\textit{1MECE1} \\
\textit{University Of Galway}\\
Galway, Ireland\\
s.knightly1@universityofgalway.ie}
\and
\IEEEauthorblockN{Seán Kelly}
\IEEEauthorblockA{\textit{1MECE1} \\
\textit{University Of Galway}\\
Galway, Ireland \\
s.kelly178@universityofgalway.ie}
}

\maketitle

\begin{abstract}
This document is a model and instructions for \LaTeX.
This and the IEEEtran.cls file define the components of your paper [title, text, heads, etc.]. *CRITICAL: Do Not Use Symbols, Special Characters, Footnotes, 
or Math in Paper Title or Abstract.
\end{abstract}

\begin{IEEEkeywords}
component, formatting, style, styling, insert.
\end{IEEEkeywords}

\section{Introduction}
Blah Blah

\subsection{Problem Statement}
Statement

\subsection{Need for IoT system}
Wearable sensors and cloud integration enables scalable real time monitoring.


\subsection{Aim and Objectives}\label{AA}
Acquire inertial data (accel + gyro) using Feathersense board.

Transmit via BLE to IoT gateway.

Perform preprocessing (feature extraction: jerk, magnitude, etc.) at the gateway.

Prepare pipeline for ML in Deliverable II.

\section{System Design}


\subsection{Block diagram of the IoT system}
Sensor Node (Feathersense: accelerometer, gyroscope, BLE)

Communication layer (BLE GATT protocol)

IoT Gateway (Raspberry Pi 4, receiving, preprocessing, visualization)

Cloud platform (for future phases, just note it)


\subsection{Simplified IoT architecture}

Perception Layer – Feathersense (sensors)

Network Layer – BLE (UART attempt → too slow, GATT → selected)

Edge Layer – Raspberry Pi (data reception, feature computation: jerk, orientation, etc.)

Application Layer – For Deliverable I: visualization, plots. For future: ML in cloud.


\section{Component Selections}\label{CS}

\subsection{Sensor Node}\label{SN}
Feathersense board (justification: built-in accel/gyro, BLE support, low-power).

\subsection{Gateway}\label{GW}
Raspberry Pi 4 (justification: computing capacity, BLE support, Python libraries).


\subsection{Communication}\label{COM}
BLE GATT (justification: faster, lightweight vs BLE UART).

\subsection{Power}
(Feathersense USB/battery powered, mention suitability for prototyping).

Provide comparisons if possible (e.g. why not ESP32, why GATT over UART).


\section{Prototype Design Plan}

\subsection{Subsystems}

\subsection{Integration Plan}

\subsection{Verification Plan}

\section{Implementation}
\subsection{Feathersense Node Setup}
how you collected accel/gyro.
\subsection{BLE Protocol}
initial trial with UART + CBOR (limitations), decision to switch to GATT.
\subsection{Data Transmission}
sampling frequency, packet size, latency.

Include screenshots/plots of transmitted data (e.g. raw accel/gyro traces).


\section{Implementation of Gateway Device}
\subsection{Data collection on Raspberry Pi}
connection with Feathersense via BLE GATT.

\subsection{Streaming Visualisation}
 live plotting of received sensor data.

(Screenshots/graphs of plots go here.)

\section{Edge/Fog Processing}
Jerk calculation (derivative of acceleration).

Magnitude of accel/gyro.

Other features relevant for ML (you can mention RMS, variance, etc. if planned).

Show small plots of raw vs processed features.


\subsection{Discussion \& Conclusion}
Summary: successful acquisition, transmission, preprocessing.

Limitations: still simulated/early prototype, real-world testing needed.

Next steps (for Deliverable II): deploy ML in cloud, integrate with Power BI.


\section*{References}

Cite like so \cite{b6}.

\begin{thebibliography}{00}
\bibitem{b1} G. Eason, B. Noble, and I. N. Sneddon, ``On certain integrals of Lipschitz-Hankel type involving products of Bessel functions,'' Phil. Trans. Roy. Soc. London, vol. A247, pp. 529--551, April 1955.
\bibitem{b2} J. Clerk Maxwell, A Treatise on Electricity and Magnetism, 3rd ed., vol. 2. Oxford: Clarendon, 1892, pp.68--73.
\bibitem{b3} I. S. Jacobs and C. P. Bean, ``Fine particles, thin films and exchange anisotropy,'' in Magnetism, vol. III, G. T. Rado and H. Suhl, Eds. New York: Academic, 1963, pp. 271--350.
\bibitem{b4} K. Elissa, ``Title of paper if known,'' unpublished.
\bibitem{b5} R. Nicole, ``Title of paper with only first word capitalized,'' J. Name Stand. Abbrev., in press.
\bibitem{b6} Y. Yorozu, M. Hirano, K. Oka, and Y. Tagawa, ``Electron spectroscopy studies on magneto-optical media and plastic substrate interface,'' IEEE Transl. J. Magn. Japan, vol. 2, pp. 740--741, August 1987 [Digests 9th Annual Conf. Magnetics Japan, p. 301, 1982].
\bibitem{b7} M. Young, The Technical Writer's Handbook. Mill Valley, CA: University Science, 1989.
\bibitem{b8} D. P. Kingma and M. Welling, ``Auto-encoding variational Bayes,'' 2013, arXiv:1312.6114. [Online]. Available: https://arxiv.org/abs/1312.6114
\bibitem{b9} S. Liu, ``Wi-Fi Energy Detection Testbed (12MTC),'' 2023, gitHub repository. [Online]. Available: https://github.com/liustone99/Wi-Fi-Energy-Detection-Testbed-12MTC
\bibitem{b10} ``Treatment episode data set: discharges (TEDS-D): concatenated, 2006 to 2009.'' U.S. Department of Health and Human Services, Substance Abuse and Mental Health Services Administration, Office of Applied Studies, August, 2013, DOI:10.3886/ICPSR30122.v2
\bibitem{b11} K. Eves and J. Valasek, ``Adaptive control for singularly perturbed systems examples,'' Code Ocean, Aug. 2023. [Online]. Available: https://codeocean.com/capsule/4989235/tree
\end{thebibliography}

\end{document}
